% !TeX spellcheck = ru_RU
%pdflatex, utf8
\documentclass[unicode, 10pt, a5paper, oneside]{article}

% Установка полей страницы
%\usepackage{anysize}
%\marginsize{0.3cm}{0.3cm}{0.3cm}{0.3cm}
\usepackage[a5paper, margin=0.3cm, bindingoffset=0cm]{geometry}

% Поддержка русского языка
\usepackage[T2A]{fontenc}		% Корректная кодировка шрифта при использовании cm-super
\usepackage[utf8]{inputenc}		% Кодировка ввода
\usepackage[russian]{babel}		% Словарь расстановки переносов
%\usepackage{cmap}				% Перекодировка символов в pdf при использовании обычного cm

% Всякие математические фишки
\usepackage{amsmath}
\usepackage{amsfonts}
\usepackage{amssymb}

% Изменение цвета, работа с графикой
\usepackage{color}
\usepackage[pdftex]{graphicx}
\graphicspath{{images/}}

% Команда для вставки ссылок \url{URL}
\usepackage[hyphens]{url}
\urlstyle{rm}					% Стиль шрифта ссылок: с засечками

% Кликабельные ссылки внутри документа
\usepackage[unicode]{hyperref}

% Включает отступ у первого абзаца в разделе
\usepackage{indentfirst}

% Настрйока стиля списков
\usepackage{enumitem}
\setlist{noitemsep, leftmargin=*, labelindent=\parindent, topsep=0pt, parsep=0pt, partopsep=0pt}

\setlist[itemize,1]{label=$\diamond$}
\setlist[itemize,2]{label=\textendash}
\setlist[itemize,3]{label=$\star$}

\renewcommand{\alph}[1]{\asbuk{#1}} % Костыль для кирилической нумерации вместо латинской
\setlist[enumerate,1]{label=\arabic*)}
\setlist[enumerate,2]{label=\alph*)}
\setlist[enumerate,3]{label=(\arabic*)}


\usepackage{textcomp}			% Команды для вставки разных символов (градусы, проценты, итд)
\usepackage{float}				% Размещение плавающих объектов там где они созданы (X)
\usepackage{wrapfig}			% Обтекаемые текстом рисунки

% Подписи у флоатов
\setlength{\intextsep}{0pt} % Отстут вокруг плавающих окружений
\usepackage{caption}
\captionsetup{parskip=0pt}
\captionsetup[figure]{labelsep=period,justification=centering,singlelinecheck=false,textfont=small,labelfont=small,aboveskip=2pt,belowskip=0pt}

% Изменение формата заголовков разделов
\usepackage{titlesec}
\titleformat{\section}{\newpage\small\bfseries}{\thesection. }{0pt}{}{}
\titlespacing*{\section}{0pt}{0pt}{0pt}

\titleformat{\subsection}{\small\bfseries}{\thesubsection. }{0pt}{}{}
\titlespacing*{\subsection}{0pt}{0pt}{0pt}

\usepackage{array}				% Позволяет объявить свои типы колонок
\usepackage{calc}				% Математика, исп-ся для расчёта ширины колонки
\usepackage{longtable}			% Длинные таблицы

% Минимальный отступ в таблицах
\setlength{\tabcolsep}{1.5mm}

% Новые типы колонок. Ширина задётся как доля от linewidth
\newcolumntype{L}[1]{p{#1\linewidth-2\tabcolsep-2\arrayrulewidth}}
\newcolumntype{C}[1]{>{\centering}p{#1\linewidth-2\tabcolsep-2\arrayrulewidth}}
\newcolumntype{R}[1]{>{\raggedleft}p{#1\linewidth-2\tabcolsep-2\arrayrulewidth}}
\newcolumntype{U}[2]{p{#1\linewidth-(#2)}}

% Стараться не оставлять одиноких строк в начале и конце абзаца
\clubpenalty=1000
\widowpenalty=1000

% Расстановка отступов и переносов
\emergencystretch=2.5em			% Максимальный промежуток между словами
\tolerance=2000
\frenchspacing
